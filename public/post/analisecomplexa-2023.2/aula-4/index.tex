---
# Documentation: https://wowchemy.com/docs/managing-content/

title: "Análise complexa - aula 4"
subtitle: "Notas por Pedro Igor"
summary: ""
authors: []
tags: []
categories: []
date: 2023-08-29T20:17:00-03:00
lastmod: 2023-08-29T20:17:00-03:00
featured: false
draft: false
math: mathjax
# Featured image
# To use, add an image named `featured.jpg/png` to your page's folder.
# Focal points: Smart, Center, TopLeft, Top, TopRight, Left, Right, BottomLeft, Bottom, BottomRight.
image:
  caption: ""
  focal_point: ""
  preview_only: false

# Projects (optional).
#   Associate this post with one or more of your projects.
#   Simply enter your project's folder or file name without extension.
#   E.g. `projects = ["internal-project"]` references `content/project/deep-learning/index.md`.
#   Otherwise, set `projects = []`.
projects: []
---


<h2 id="curvas-continuação">Curvas (continuação)</h2>
<div class="Def">
<p><strong>Definição 1</strong>. <em>A equivalência de curvas, definida
na aula passada é, de fato, uma relação de equivalência e uma curva é
uma classe de equivalência de curvas parametrizadas.</em></p>
</div>
<div class="Def">
<p><strong>Definição 2</strong>. <em>Dada uma curva parametrizada <span
class="math inline">\(\alpha : [a,b] \rightarrow \mathbb{C}\)</span>, a
curva "invertida" é a curva <span class="math inline">\(\alpha^{-}:[a,b]
\rightarrow \mathbb{C}\)</span> dada por <span
class="math inline">\(\alpha^{-} = \alpha(b+a-t).\)</span></em></p>
</div>
<div class="Def">
<p><strong>Definição 3</strong>. <em>Dizemos que a curva é fechada
quando <span class="math inline">\(\alpha(a) =
\alpha(b).\)</span></em></p>
</div>
<div class="Def">
<p><strong>Definição 4</strong>. <em>A curva é dita simples se <span
class="math inline">\(\alpha(t_1)\neq \alpha (t_2)\)</span> sempre que
<span class="math inline">\(t_1 \neq t_2\)</span>.</em></p>
</div>
<p>Não-exemplo:</p>
<div class="center">

</div>
<div class="Def">
<p><strong>Definição 5</strong>. <em>Uma curva é fechada e simples é uma
curva <span class="math inline">\(\alpha: [a,b] \rightarrow
\mathbb{C}\)</span> fechada tal que se <span
class="math inline">\(\alpha(t_1) = \alpha (t_2) \Rightarrow
t_1=t_2\)</span> ou <span class="math inline">\(\{t_1,t_2\} =
\{a,b\}\)</span>.</em></p>
</div>
<p>Para simplificar nos referiremos como <span
class="math inline">\(\textbf{curva}\)</span> uma curva parametrizada
<span class="math inline">\(C^1\)</span>.<br />
Exemplo: circunferência. <span class="math inline">\(C_r(z_0) = \{z \in
\mathbb{C};|z-z_0|=r\)</span>. Cuja paramtetrização é <span
class="math inline">\(z = z_0 + e^{it}, t \in [0,2\pi]\)</span>.Dado um
círculo <span class="math inline">\(C = C_r(z_0),\)</span> diremos que a
parametrização acima, ou qualquer outra equivalente, é positiva enquanto
a parametrização inversa é negativa.</p>
<h2 id="integral-de-curva">Integral de Curva</h2>
<div class="Def">
<p><strong>Definição 6</strong>. <em>Seja <span
class="math inline">\(\gamma\)</span> uma curva e <span
class="math inline">\(\alpha\)</span> uma parametrização, definimos a
integral de <span class="math inline">\(f\)</span> sobre <span
class="math inline">\(\gamma\)</span>, ou ao longo de <span
class="math inline">\(\gamma\)</span>, da seguinte maneira: <span
class="math display">\[\int_{\gamma}f(z)dz =
\int_{a}^{b}f(\alpha(t))\alpha&#39;(t)dt.\]</span></em></p>
</div>
<p>Observe que essa definição não depende da parametrização. De fato, se
<span class="math inline">\(\beta : [c,d] \rightarrow C\)</span> é outra
parametrizaç˜ão de <span class="math inline">\(\gamma\)</span> e <span
class="math inline">\(t: [c,d] \rightarrow [a,b]\)</span> é tal que
<span class="math inline">\(\alpha(t(s)) = \beta(s)\)</span>, então
<span class="math display">\[\int_{c}^df(\beta(s)) \cdot \beta
&#39;(s)ds = \int_c^df(\alpha(t(s))) \cdot \alpha &#39;(t(s)) \cdot
t&#39;(s) ds = \int_a^b f(\alpha(t)) \cdot \alpha &#39;(t)dt .\]</span>
Além disso se <span
class="math inline">\(\{a_0,a_1,a_2,...,a_n\}\)</span> é uma partição de
<span class="math inline">\([a,b]\)</span> tal que <span
class="math inline">\(\alpha\)</span> é <span
class="math inline">\(C^1\)</span> em <span
class="math inline">\([a,b]\)</span>, então <span
class="math display">\[\int_{\gamma}f(z)dz = \sum_{i =
0}^{n-1}\int_{a_i}^{a_{i+1}}f(\alpha(t)) \cdot dt.\]</span></p>
<div class="Def">
<p><strong>Definição 7</strong>. <em>Dada uma curva <span
class="math inline">\(\gamma\)</span> e <span
class="math inline">\(\alpha : [a,b] \rightarrow \mathbb{C}\)</span> uma
parametrização, o comprimento da curva é dado por <span
class="math display">\[l(\gamma).=.\int_a^b|\alpha&#39;(t)|dt.\]</span><br />
</em></p>
</div>
<div class="ex">
<p><strong>Exercício 1</strong>. <em>Mostre que <span
class="math inline">\(l(\gamma)\)</span> está bem definido, ou seja,
independe da parametrização escolhida.</em></p>
</div>
<div class="prop">
<p><strong>Proposição 1</strong>. <em>Sejam <span
class="math inline">\(f,g\)</span> contínuas na imagem de <span
class="math inline">\(\gamma\)</span>. Então,</em></p>
<ol>
<li><p><em><span class="math inline">\(\forall w_1,w_2 \in
\mathbb{C}\)</span>,temos <span
class="math display">\[\int_{\gamma}(w_1f(z)+w_2g(z))dz = w_1 \cdot
\int_{\gamma}f(z)dz+w_2 \cdot \int_{\gamma}g(z)dz\]</span></em></p></li>
<li><p><em>se <span class="math inline">\(\gamma^-\)</span> é a curva
inversa de <span class="math inline">\(\gamma\)</span>, então <span
class="math display">\[\int_{\gamma^-}f(z)dz = -
\int_{\gamma}f(z)dz\]</span></em></p></li>
<li><p><em>vale a desigualdade: <span class="math display">\[\left|
\int_{\gamma}f(z)dz \right| \leqslant l(\gamma) \cdot \sup_{z \in
\gamma}|f(z)|\]</span></em></p></li>
</ol>
<p><em>Prova: Segue das propriedades básicas de integral de funções
contínuas em intervalos de <span
class="math inline">\(\mathbb{R}\)</span>.</em></p>
</div>
<div class="theorem">
<p><strong>Teorema 1</strong>. <em>Se <span class="math inline">\(\Omega
\subseteq \mathbb{C}\)</span> é um domínio e <span
class="math inline">\(f: \Omega \rightarrow \mathbb{C}\)</span> é
contínua e possui primitiva F, então <span
class="math display">\[\int_{\gamma}f(z)dz = F(\alpha(b)) -
F(\alpha(a)),\]</span> onde <span class="math inline">\(\alpha\)</span>
é qualquer parametrização de <span
class="math inline">\(\gamma.\)</span></em></p>
</div>
<p>Prova: tome <span class="math inline">\(F&#39;=f,\)</span> <span
class="math display">\[\int_{\gamma}f(z)dz = \int_a^bf(\alpha(t)) \cdot
\alpha &#39;(t)dt = \int_a^bF&#39;(\alpha(t)) \cdot \alpha
&#39;(t)dt\]</span> <span class="math display">\[\Rightarrow^*
\int_{\gamma}f(z)dz = \int_a^b(F \circ \alpha)&#39;(t)dt = F(\alpha(b))
- F(\alpha(a)).\]</span> Note que a implicação <span
class="math inline">\(*\)</span> segue da regra da cadeia para funções
em <span class="math inline">\(\mathbb{R}^n\)</span> e do fato que F é
holomorfa <span class="math inline">\(\Rightarrow\)</span> F
difereniável do ponto de vista de <span
class="math inline">\(\mathbb{R}^2.\)</span><br />
</p>
<div class="obs">
<p><strong>OBS 1</strong>. <em>Por simplificação, as vezes escreveremos
<span class="math display">\[\int_{\gamma}f(z)dz = F(\gamma(b)) -
F(\gamma(a)),\]</span></em></p>
</div>
<div class="corollary">
<p><strong>Corolário 1</strong>. <em>Sob as mesmas hipóteses do teorema
acima , se <span class="math inline">\(\gamma\)</span> for fechada,
então <span class="math display">\[\int_{\gamma}f(z)dz =
0.\]</span></em></p>
</div>
<p>Considere o seguinte exemplo. Seja <span class="math inline">\(C =
C_r(0)\)</span> uma circunferência ao redor da origem com orientação
positiva e <span class="math inline">\(f(z) = \dfrac{1}{z}\)</span>.
Então, <span class="math display">\[\int_{\gamma}f(z)dz =
\int_0^{2\pi}\dfrac{1}{re^{it}} \cdot ire^{it}dt = i \int_0^{2\pi} =
2\pi i \neq 0.\]</span> Logo, se <span
class="math inline">\(\Omega\)</span> é um domínio contendo <span
class="math inline">\(C_r(0)\)</span>, então <span
class="math inline">\(\dfrac{1}{z}\)</span> não possui primitiva em
<span class="math inline">\(\Omega.\)</span></p>
<div class="corollary">
<p><strong>Corolário 2</strong>. <em>Se <span
class="math inline">\(\Omega\)</span> é um domínio e <span
class="math inline">\(f : \Omega \rightarrow \mathbb{C}\)</span> é
holomorfa é tal que <span class="math inline">\(f&#39;(z) = 0 ,\forall z
\in \Omega\)</span>. Então, <span class="math inline">\(f =\)</span>
constante.<br />
</em></p>
</div>
<div class="proof">
<p><em>Proof.</em> Como <span class="math inline">\(\Omega\)</span> é
conexo pelo exercício 5 do capitulo 1 do livro do Stein, <span
class="math inline">\(\Omega\)</span> é conexo por arcos. Isto é, dados
<span class="math inline">\(z_1,z_2 \in \Omega, \exists \alpha : [a,b]
\rightarrow \mathbb{C}\)</span> uma curva parametrizada tal que <span
class="math inline">\(\alpha(a) = z_1, \alpha(b) = z_2\)</span> e <span
class="math inline">\(\alpha(t) \in \Omega \forall t \in [a,b].\)</span>
Com isso, <span class="math inline">\(f(z_2) - f(z_1) = f(\alpha(a)) -
f(\alpha(b)) = \int_{\gamma}f&#39;(z)dz = 0,\)</span> onde <span
class="math inline">\(\gamma\)</span> é a curva associada à
parametrização <span class="math inline">\(\alpha.\)</span> ◻</p>
</div>
<h2 id="o-teorema-de-cauchy">O Teorema de Cauchy</h2>
<div class="theorem">
<p><strong>Teorema 2</strong>. <em>Se <span
class="math inline">\(f\)</span> é holomorfa em um domínio <span
class="math inline">\(\Omega\)</span> e <span
class="math inline">\(\gamma\)</span> é uma curva fechada cujo <span
class="math inline">\(\textbf{interior}\)</span> está contido em <span
class="math inline">\(\Omega\)</span>, então <span
class="math display">\[\int_{\gamma}f(z)dz = 0.\]</span><br />
Observações:</em></p>
<ul>
<li><p><em>Ao exigir que o interior da curva esteja no conjunto, não
precisamos exigir que <span class="math inline">\(f\)</span> possua
primitiva;</em></p></li>
<li><p><em>Isso não contradiz o exemplo que vimos envolvendo <span
class="math inline">\(C_r(0)\)</span> e <span class="math inline">\(f(z)
= \frac{1}{z}\)</span> pois, f não é holomorfa em <span
class="math inline">\(z = 0\)</span> e 0 está no interior do
círculo.</em></p></li>
</ul>
</div>
<p>Vamos começar com uma versão simplificada do teorema.</p>
<div class="theorem">
<p><strong>Teorema 3</strong> (Goursat). <em>Se <span
class="math inline">\(T \subseteq \Omega\)</span> é um triângulo cujo
interior está contido em <span class="math inline">\(\Omega\)</span>
então <span class="math display">\[\int_{T}f(z)dz = 0.\]</span></em></p>
</div>
<div class="Def">
<p><strong>Definição 8</strong>. <em>Triângulo. Seja <span
class="math inline">\(\{a_0,a_1,a_2,a_3\}\)</span> partição de <span
class="math inline">\([a,b]\)</span>. Tome <span
class="math inline">\(\alpha: [a,b]\rightarrow \mathbb{C},\)</span> tal
que <span class="math display">\[\alpha(a_0)=A,\quad \alpha(a_1) =
B,\quad \alpha(a_2) = C,\quad \alpha(a_3)=A\]</span> e, além disso,
<span class="math display">\[\alpha([a_0,a_1]) = \overline{AB},\quad
\alpha([a_1,a_2]) = \overline{BC}\quad \alpha([a_2a_3]) =
\overline{CA}\]</span> as respectivas parametrizações dos
segmentos.</em></p>
<div class="center">

</div>
</div>
<div class="proof">
<p><em>Proof.</em> Chamaremos o triângulo original de <span
class="math inline">\(T=T_0\)</span> e usando os pontos médios <span
class="math inline">\(E,D\)</span> e <span
class="math inline">\(F\)</span> dividiremos <span
class="math inline">\(T_0\)</span> em quatro triiângulos <span
class="math inline">\(T_1^1 = \triangle AEF, T_1^2 = \triangle FBD,
T_1^3 = \triangle CED\)</span> e <span class="math inline">\(\triangle
T_1^4 = DEF\)</span>, todos semelhantes ao original com um fator de
semelhança igual a <span
class="math inline">\(\dfrac{1}{2}\)</span>.</p>
<div class="center">

</div>
<p>Parametrizando os triângulos no sentido anti-horário temos que <span
class="math display">\[\int_{T_0}f(z)dz = \int_{T_1^1}f(z)dz+
\int_{T_1^2}f(z)dz + \int_{T_1^3}f(z)dz + \int_{T_1^4}f(z)dz\]</span>
Portanto, <span class="math display">\[\left| \int_{T_0}f(z)dz \right|
\leqslant \sum_{i=1}^4\left| \int_{T_1^i}f(z)dz \right| \leqslant 4
\cdot \max_{1 \leqslant i \leqslant 4} \left| \int_{T_1^i}f(z)dz
\right|.\]</span> Seja <span class="math inline">\(i\)</span> tal que
<span class="math inline">\(T_1^i\)</span> realiza o máximo acima,
defina <span class="math inline">\(T_1 = T_1^i\)</span>. Ou seja, temos,
<span class="math display">\[\left| \int_{T_0} f(z)dz \right| \leqslant
4 \left| \int_{T_1}f(z)dz \right|.\]</span> Repetindo o processo,
encontramos <span class="math inline">\(T_2\)</span>, contido no
interior de <span class="math inline">\(T_1\)</span>, demelhante a <span
class="math inline">\(T_1\)</span>, com fator de semelhança igual a
<span class="math inline">\(\dfrac{1}{2}\)</span> e tal que <span
class="math display">\[\left| \int_{T_1}f(z)dz \right| \leqslant 4 \cdot
\left| \int_{T_2}f(z)dz \right|.\]</span> Iterando o processo, temos uma
sequência de triângulos <span class="math inline">\(\{T_n\}\)</span>
tais que o fator de semelhança entre eles é <span
class="math inline">\(\dfrac{1}{2}\)</span> e <span
class="math display">\[\left| \int_{T_{n}}f(z)dz \right| \leqslant 4
\cdot \left| \int_{T_{n+1}}f(z)dz \right|.\]</span> Sejam <span
class="math inline">\(p_n =\)</span> perímetro de <span
class="math inline">\(T_n\)</span> e <span class="math inline">\(d_n =
\operatorname{diam}(T_n).\)</span> Por conta do fator de semelhança,
segue que <span class="math inline">\(p_{n+1} = \dfrac{1}{2}\)</span> e
<span class="math inline">\(d_{n+1} = \dfrac{1}{2}d_{n}.\)</span>
Concatenando os resultados, temos que <span class="math inline">\(d_n =
\dfrac{1}{2^n}d_0\)</span> , <span class="math inline">\(p_n =
\dfrac{1}{2^n}p_0\)</span> e <span class="math display">\[\left|
\int_{T_0}f(z)dz \right| \leqslant 4^n \cdot \left| \int_{T_n}f(z)dz
\right|.\]</span> Seja <span
class="math inline">\(\widetilde{T_n}\)</span> = fecho do interior de
<span class="math inline">\(T_n.\)</span> Então: <span
class="math inline">\(\widetilde{T_n}\)</span> é compacto, <span
class="math inline">\(\operatorname{diam}(\widetilde{T_n}) =
\operatorname{diam}(T_n)\)</span> e <span
class="math inline">\(\widetilde{T_{n+1}} \subseteq \widetilde{T_n}
,\forall n \in \mathbb{N}.\)</span> Logo, pelo teorema dos compactos
encaixados, existe um único ponto <span
class="math inline">\(z_0\)</span> que está contido em todos os <span
class="math inline">\(T_n\)</span>. Como <span
class="math inline">\(f\)</span> é holomorfa em <span
class="math inline">\(z_0\)</span>, dado <span
class="math inline">\(\epsilon &gt; 0 , \exists \delta &gt; 0\)</span>
tal que se <span class="math inline">\(|z-z_0|&lt; \delta\)</span> então
<span class="math inline">\(\left|
\dfrac{f(z)-f(z_0)}{z-z_0}-f&#39;(z_0) \right| &lt; \epsilon .\)</span>
Logo, <span class="math display">\[|f(z)-f(z_0)-f&#39;(z_0)(z-z_0)| &lt;
\epsilon |z-z_0|.\]</span> Tome <span class="math inline">\(n\)</span>
grande suficiente de modo que <span class="math inline">\(d_n &lt;
\delta.\)</span> Temos neste caso, que <span
class="math inline">\(|z-z_0|&lt; \delta| , \forall z \in T_n\)</span>.
Logo, temos <span class="math display">\[\left| \int_{T_n}[
f(z)-f(z_0)-f&#39;(z_0)(z-z_0)]dz \right| &lt; \epsilon
\int_{T_n}|z-z_0|dz &lt; \epsilon \cdot d_n \cdot p_n = \dfrac{\epsilon
d_0 p_0}{4^n}.\]</span> Uma vez que a função constante <span
class="math inline">\(f(z_0)\)</span> e a função <span
class="math inline">\(f&#39;(z_0) (z-z_0)\)</span> possuem primitiva em
<span class="math inline">\(\mathbb{C}\)</span>, segue que <span
class="math display">\[\int_{\gamma}f(z_0)dz =
\int_{\gamma}f&#39;(z_0)(z-z_0)dz = 0.\]</span> Portanto, temos que
<span class="math display">\[\left| \int_{T_n}f(z)dz \right|&lt;
\dfrac{\epsilon d_0 p_0}{4^n}.\]</span> Disto segue que <span
class="math display">\[\left| \int_{T_0}f(z)dz \right| \leqslant 4^n
\cdot \dfrac{\epsilon d_0 p_0}{4^n} = \epsilon d_0 p_0 .\]</span> Como
<span class="math inline">\(\epsilon\)</span> é arbitrário, o resultado
segue. ◻</p>
</div>

